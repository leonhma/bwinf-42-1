\documentclass[a4paper,12pt]{scrartcl}

\usepackage[english,ngerman]{babel}
\usepackage{csquotes}
\usepackage{color}
\usepackage{siunitx}
\usepackage{amssymb}
\usepackage{amsthm}
\usepackage{amsmath}
\usepackage{graphicx}
\usepackage{anyfontsize}
\usepackage[hidelinks]{hyperref}
\usepackage{subcaption}
\usepackage{enumitem}
\usepackage{tabu}
\usepackage{tabularx}
\usepackage[ngerman]{cleveref}
\usepackage[onehalfspacing]{setspace}
\usepackage[ backend=biber]{biblatex}
\addbibresource{refs.bib}

\title{\textbf{\Huge Aufgabe 3: Zauberschule}}
\author{\LARGE Team-ID: \LARGE 00128 \\\\
	    \LARGE Team-Name: \LARGE E29C8CF09F8E89 \\\\
	    \LARGE Bearbeiter/-innen dieser Aufgabe: \\ 
	    \LARGE Leonhard Masche\\\\}
\date{\LARGE1. September 2023}

% code
\usepackage[cache=false]{minted}
\usemintedstyle{xcode}

% set up geometry
% \usepackage[lmargin={3.5cm},rmargin={2cm}, tmargin={2cm},bmargin = {2cm}]{geometry}

% font
\usepackage{fontspec}
% \setmonofont{Fira Code SemiBold}
\setmainfont{Arial}

% configure headers and footers
\graphicspath{{./assets/}}

% custom commands
\newcommand*\wildcard[2][5cm]{\vspace*{2cm}\parbox{#1}{\hrulefill\par#2}}
\newcommand{\fig}[3][]{\begin{figure}[H] \includegraphics[#1]{#2}
\centering\caption{#3}\end{figure}}
\newcommand{\pyinline}[1]{\mintinline{python}{#1}}

\pagenumbering{arabic}

\theoremstyle{definition}
\newtheorem{theorem}{Theorem}[section]
\newtheorem{lemma}[theorem]{Lemma}

% hyphenation
\hyphenation{Fraun-hof-er}

% \setcounter{biburlnumpenalty}{100}
% \setcounter{biburlucpenalty}{100}
% \setcounter{biburllcpenalty}{100}

\usepackage[hang]{footmisc}
\setlength{\footnotemargin}{0.8em}

\begin{document}

\maketitle
\tableofcontents
\section{Lösungsidee}

Zur Lösung dieses Shortest-Path-Problems gibt es einige bekannte Algorithmen.
Die möglichen Wege in der Zauberschule können als gewichteter Graph dargestellt
werden, wobei Bewegungen in die vier Richtungen (links, oben, rechts, unten) ein
Gewicht von $1$, und Stockwerkwechsel ein Gewicht von $3$ haben.\footnote{
	Das Programm wurde entsprechend der originalen Aufgabenstellung vom 1. Sept.
	geschrieben. Das bedeutet, Distanzen werden wie im Aufgabenblatt dargestellt
	berechnet.} Um nun einen
kürzesten Pfad zu finden, wird Dijkstra's Algorithmus verwendet: Für jeden
besuchten Knoten (Feld) wird dessen Vorgänger gespeichert, sodass letztendlich
der Pfad selbst zurückverfolgt werden kann. Entsprechend Dijktra's Algorithmus
ist dies der kürzest mögliche Pfad.\footfullcite[Vgl.][]{dijkstraNoteTwoProblems1959}


\section{Umsetzung}

Das Programm (\texttt{program.py}) ist in Python umgesetzt und mit einer Umgebung ab
der Version 3.8 ausführbar. Zum Umgang mit Matrizen wird die externe
Bibliothek \texttt{numpy} verwendet.

Beim Ausführen der Datei wird zuerst nach der Zahl des Beispiels gefragt. Dieses
wird nun aus der Datei \texttt{input/zauberschule\{n\}.txt} geladen und bearbeitet. Das
Ergebnis wird, zusammen mit einigen Werten, ausgegeben. Das resultierende
Zauberschule-Gitter wird zusätzlich in eine Datei geschrieben.

\section{Beispiele}

Hier wird das Programm auf die 6 Beispiele von der Website angewendet.
Zusätzlich wird ein eigenes Beispiel (\texttt{zauberschule6.txt}) bearbeitet, welches
eine unlösbare Aufgabe darstellt:

\begin{itemize}
	\item \textbf{zauberschule0.txt}\begin{minted}[breaklines]{text}
Oben:         Unten:
############# #############
#.....#.....# #.......#...#
#.###.#.###.# #...#.#.#...#
#...#.#...#.# #...#.#.....#
###.#.###.#.# #.###.#.#.###
#...#.....#.# #.....#.#...#
#.#########.# #####.###...#
#.....#.....# #.....#.....#
#####.#.###.# #.#########.#
#....A#B..#.# #...#!>!....#
#.#########.# #.#.#.#.###.#
#...........# #.#...#...#.#
############# #############

Ausgabe gespeichert in "output/zauberschule0.txt"

Weg mit Länge 7s in 71 Iterationen (0.61ms) gefunden.
	\end{minted}
	\item \textbf{zauberschule1.txt}\begin{minted}[breaklines]{text}
Oben:                 Unten:
##################### #####################
#...#.....#...#.....# #.......#.....#.....#
#.#.#.###.#.#.#.###.# #.###.#.#.###.#.###.#
#.#.#...#.#.#...#...# #.....#.#.#.....#.#.#
###.###.#.#.#####.### #######.#.#######.#.#
#.#.#...#.#B....#...# #.....#.#.....#...#.#
#.#.#.###.#^###.##### #.###.#.#.###.###.#.#
#.#...#.#.#^<A#.....# #.#.#...#.#...#...#.#
#.#####.#.#########.# #.#.#######.###.###.#
#...................# #...........#.......#
##################### #####################

Ausgabe gespeichert in "output/zauberschule1.txt"

Weg mit Länge 2s in 4 Iterationen (0.16ms) gefunden.
	\end{minted}
	\item \textbf{zauberschule2.txt}\begin{minted}[breaklines]{text}
Oben:                                         
#############################################
#...#.....#...........#.#...........#.......#
#.#.#.###.#########.#.#.#.#######.#.#.#####.#
#.#.#...#.#.........#A>!#!#.....#.#.#...#...#
###.###.#.#.#############v#.#.###.#.###.#.###
#.#.#...#.#..............>>B#.#...#.#...#.#.#
#.#.#.###.###########.#########.###.#.###.#.#
#.#...#.#.#.........#.#.#.....#.#.....#.#.#.#
#.#####.#.#.#######.#.#.#.###.#.#######.#.#.#
#.....#...#...#.#...#...#.#.#.#.......#.#.#.#
#.#####.#####.#.#.#######.#.#.#######.#.#.#.#
#.....#.......#.#.#.....#.#.#.#...#...#...#.#
#.###.#########.#.###.#.#.#.#.#.#.#.###.###.#
#...#.................#...#.....#...#.......#
#############################################

Unten:
#############################################
#...#.......#.....#.....#...#...#.....#.....#
#.#.#.#####.###.#.###.#.#.#.#.#.#.###.###.###
#.#.#.....#.#...#.....#!>!#.#.#...#.#...#...#
###.#.###.#.#.#############.#.#####.###.###.#
#.#.#...#.#.#.#.....#...#.#.#...#.#...#.#...#
#.#.#####.#.#.#.###.#.#.#.#.#.#.#.#.#.#.#.###
#.#...#...#.#.....#.#.#...#.#.#.#...#.#.#...#
#.###.#.###.#.#####.#.###.#.###.#####.#.###.#
#...#.#.#...#...#...#...#.#...#.#.....#.....#
#.###.#.#.#######.#####.#####.#.#.#.#######.#
#...#...#...#...#.....#.......#.#.#.#.....#.#
#.#.#######.#.#.#####.#.#######.#.###.###.#.#
#.#...........#.......#.........#.......#...#
#############################################

Ausgabe gespeichert in "output/zauberschule2.txt"

Weg mit Länge 10s in 93 Iterationen (0.79ms) gefunden.
	\end{minted}
	\newpage
	\item \textbf{zauberschule3.txt}\begin{minted}[breaklines]{text}
Oben:                           Unten:
############################### ###############################
#...#.....#...........#.......# #.......#.....#...#...#.......#
#.#.#.###.#.###.#######.###.#.# #.###.#.#.#.#.#.###.#.#.#.#####
#.#.#...#.#.#.#.#.......#...#.# #.....#.#.#.#.#.....#.#.#.....#
###.###.#.#.#.#.#.#######.###.# #######.#.#.#.#.#####.#.#####.#
#.#.#...#.#...#.....#.....#.#.# #.....#.#.#.#.#.#...#...#.....#
#.#.#.###.###########.#####.#.# #.###.#.###.#.###.#.#.###.#####
#.#...#.#.#.........#...#.....# #...#.#.#...#.....#.#.#.#.....#
#.#####.#.#.#######.###.#.##### #.#.###.#.#########.#.#.#####.#
#...#.#...#...#.#...#...#.#...# #.#.....#.#.......#.#.....#...#
#.#.#.#.#####.#.#.###.###.#.#.# #.#######.#######.#.#.#####.#.#
#.#.#.#.......#.#.#...#.#...#.# #...#...#.....#...#.#.#...#.#.#
#.#.#.#########.#.#.###.#####.# ###.#.#.#.###.#.###.#.#.#.#.#.#
#.#.......#.#.....#.#.#.....#.# #.#.#.#...#...#.....#.#.#.#.#.#
#.#######.#.#.#####.#.#.###.#.# #.#.#.#####.###.#######.#.#.#.#
#.#.....#...#.#.#...#.#.#.#...# #.#.#.....#.#.....#.....#.#.#.#
#.#.###.#####.#.#.#.#.#.#.##### #.#.#######.#####.###.###.#.#.#
#.#...#.....#.#.#.#.#...#.....# #.#.....#...#...#.....#.#...#.#
#.###.#####.#.#.#.#.###.#####.# #.#####.#.###.#.#######.#####.#
#...#.#...#...#...#...#.......# #.#...#.#.#...#.......#.#.....#
#.#.#.#.#############.#.####### #.#.#.#.#.#.#########.#.#.#####
#.#.#.#...............#.#.....# #...#.#.#.#...#...#.#.#.#.....#
#.###.#.#######.#########.###.# #####.#.#.###.#.#.#.#.#.#####.#
#...#.#.#...#...#.........#...# #.....#.#.......#.#.#.#...#...#
###.#.###.#.#.###.#####.###.#.# #.###.#.#########.#.#.#.#.#.###
#...#.....#.#.....#>>B#.#...#.# #...#.#.....#.#.....#.#.#.#.#.#
#.#########.#######^#.###.###.# #.#.#######.#.#.#####.#.#.#.#.#
#..A#>>>>v#.#>>>>>>^#...#.#.#.# #.#.......#.#...#.....#.#...#.#
#.#v#^###v#.#^#########.#.#.#.# #.###.#####.#####.#####.#####.#
#.#>>^..#>>>>^#...........#...# #...#.............#...........#
############################### ###############################

Ausgabe gespeichert in "output/zauberschule3.txt"

Weg mit Länge 14s in 218 Iterationen (1.78ms) gefunden.
	\end{minted}
	\item \textbf{zauberschule4.txt}\begin{minted}[breaklines]{text}
Ausgabe gespeichert in "output/zauberschule4.txt"

Weg mit Länge 51s in 7043 Iterationen (45.75ms) gefunden.
	\end{minted}
	\item \textbf{zauberschule5.txt}\begin{minted}[breaklines]{text}
Ausgabe gespeichert in "output/zauberschule5.txt"

Weg mit Länge 75s in 10791 Iterationen (65.23ms) gefunden.
			\end{minted}
	\item \textbf{zauberschule6.txt}\begin{minted}[breaklines]{text}
ValueError: Es wurde kein Pfad gefunden!
			\end{minted}
\end{itemize}

\section{Quellcode}

\textit{program.py}
\inputminted[breaklines,linenos,fontsize=\footnotesize]{python3}{program.py}
\newpage

\end{document}