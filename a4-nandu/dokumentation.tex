\documentclass[a4paper,12pt]{scrartcl}

\usepackage[english,ngerman]{babel}
\usepackage{csquotes}
\usepackage{color}
\usepackage{siunitx}
\usepackage{amssymb}
\usepackage{amsthm}
\usepackage{amsmath}
\usepackage{graphicx}
\usepackage{anyfontsize}
\usepackage[hidelinks]{hyperref}
\usepackage{subcaption}
\usepackage{enumitem}
\usepackage{tabu}
\usepackage{tabularx}
\usepackage[ngerman]{cleveref}
\usepackage[onehalfspacing]{setspace}
%\usepackage[ backend=biber]{biblatex}
%\addbibresource{refs.bib}

\title{\textbf{\Huge Aufgabe 4: „Nandu“}}
\author{\LARGE Team-ID: \LARGE 00128 \\\\
	    \LARGE Team-Name: \LARGE E29C8CF09F8E89 \\\\
	    \LARGE Bearbeiter/-innen dieser Aufgabe: \\ 
	    \LARGE Leonhard Masche\\\\}
\date{\LARGE20. Oktober 2023}

% code
\usepackage[cache=false]{minted}
\usemintedstyle{xcode}

% set up geometry
% \usepackage[lmargin={3.5cm},rmargin={2cm}, tmargin={2cm},bmargin = {2cm}]{geometry}

% font
\usepackage{fontspec}
% \setmonofont{Fira Code SemiBold}
\setmainfont{Arial}

% configure headers and footers
\graphicspath{{./assets/}}

% custom commands
\newcommand*\wildcard[2][5cm]{\vspace*{2cm}\parbox{#1}{\hrulefill\par#2}}
\newcommand{\fig}[3][]{\begin{figure}[H] \includegraphics[#1]{#2}
\centering\caption{#3}\end{figure}}
\newcommand{\pyinline}[1]{\mintinline{python}{#1}}

\pagenumbering{arabic}

\theoremstyle{definition}
\newtheorem{theorem}{Theorem}[section]
\newtheorem{lemma}[theorem]{Lemma}

% hyphenation
\hyphenation{Fraun-hof-er}

% \setcounter{biburlnumpenalty}{100}
% \setcounter{biburlucpenalty}{100}
% \setcounter{biburllcpenalty}{100}

\usepackage[hang]{footmisc}
\setlength{\footnotemargin}{0.8em}

\begin{document}

\maketitle
\tableofcontents
\section{Lösungsidee}

Die Bausteine aus der Aufgabe können leicht als Lambda-Funktionen modelliert
werden. Die Aufgabe wird als Matrix geladen und es wird über die einzelnen Zeilen
und Buchstaben iteriert um Bausteine zu finden. Licht-Zustände werden in einer
eigenen Matrix gespeichert, in der zu Anfang die Eingabezustände eingetragen
werden. Wenn beim Iterieren zwei Buchstaben gefunden werden, die zu einem
bekannten Baustein passen, so werden die Eingabezustände des Bausteins aus der
Licht-Zustands-Matrix geladen und die Ausgabe mithilfe der Lambda-Funktion des
Baustein errechnet, welche dann wiederum in die Licht-Zustands-Matrix eingefügt
wird. Nachdem ein Baustein bearbeitet wurde, wird (wenn möglich) gleich zwei
Felder nach rechts gesprungen, um ein wiederholtes Anwenden eines Bausteins zu
vermeiden. Dies wird über alle Zeilen der Konstruktion (bis auf die Letzte mit
ausschließlich Ausgabe-Lampen) fortgeführt. Zuletzt wird das Ergebnis an den
Ausgabe-Lampen ausgelesen. Um alle möglichen Eingaben für ein Konstrukt zu
simulieren gibt es bei $n$ Eingabe-Lampen $n^2$ Möglichkeiten für
unterschiedliche Ausgaben, welche alle durch den vorher genannten Prozess
simuliert und in einer Tabelle notiert werden.


\section{Umsetzung}

Das Programm (\texttt{program.py}) ist in Python umgesetzt und mit einer Umgebung ab
der Version 3.6 ausführbar. Zum Umgang mit Matrizen wird die externe
Bibliothek \texttt{numpy}, für die Verwendung von Tabellen \texttt{pandas} verwendet. Alle
Vorraussetzungen für das Ausführen des Programmes können mit dem Befehl \texttt{pip
	install -r requirements.txt} installiert werden.

Beim Ausführen der Datei wird zuerst nach der Zahl des Beispiels gefragt. Dieses
wird nun aus der Datei \texttt{input/zauberschule\{n\}.txt} geladen und bearbeitet. Nun
werden durch einen einfachen binären Zähler alle Eingabezustände simuliert und
das Ergebnis ausgegeben. Zusätzlich wird es auch als \texttt{csv}-Datei im Ordner
\texttt{output} gespeichert, was eine programmatische Verifizierung der Ergebnisse
erleichtert.

\section{Beispiele}

Hier wird das Programm auf die 5 Beispiele von der Website, sowie das linke
Beispiel aus der Aufgabenstellung (\texttt{nandu0.txt}) angewendet:

\begin{itemize}
	\item \textbf{nandu0.txt}\begin{minted}[breaklines]{text}
Simuliert in 0.80ms

    Q1   Q2   L1   L2
0  Aus  Aus  Aus  Aus
1  Aus   An  Aus  Aus
2   An  Aus  Aus  Aus
3   An   An   An   An

Ausgabe gespeichert in "output/nandu0.csv"
	\end{minted}
	\item \textbf{nandu1.txt}\begin{minted}[breaklines]{text}
Simuliert in 0.80ms

    Q1   Q2   L1   L2
0  Aus  Aus   An   An
1  Aus   An   An   An
2   An  Aus   An   An
3   An   An  Aus  Aus

Ausgabe gespeichert in "output/nandu1.csv"
	\end{minted}
	\item \textbf{nandu2.txt}\begin{minted}[breaklines]{text}
Simuliert in 1.09ms

    Q1   Q2   L1   L2
0  Aus  Aus  Aus   An
1  Aus   An  Aus   An
2   An  Aus  Aus   An
3   An   An   An  Aus

Ausgabe gespeichert in "output/nandu2.csv"
	\end{minted}
	\item \textbf{nandu3.txt}\begin{minted}[breaklines]{text}
Simuliert in 1.60ms

    Q1   Q2   Q3   L1   L2   L3   L4
0  Aus  Aus  Aus   An  Aus  Aus   An
1  Aus  Aus   An   An  Aus  Aus  Aus
2  Aus   An  Aus   An  Aus   An   An
3  Aus   An   An   An  Aus   An  Aus
4   An  Aus  Aus  Aus   An  Aus   An
5   An  Aus   An  Aus   An  Aus  Aus
6   An   An  Aus  Aus   An   An   An
7   An   An   An  Aus   An   An  Aus

Ausgabe gespeichert in "output/nandu3.csv"
	\end{minted}
	\item \textbf{nandu4.txt}\begin{minted}[breaklines]{text}
Simuliert in 1.95ms

     Q1   Q2   Q3   Q4   L1   L2
0   Aus  Aus  Aus  Aus  Aus  Aus
1   Aus  Aus  Aus   An  Aus  Aus
2   Aus  Aus   An  Aus  Aus   An
3   Aus  Aus   An   An  Aus  Aus
4   Aus   An  Aus  Aus   An  Aus
5   Aus   An  Aus   An   An  Aus
6   Aus   An   An  Aus   An   An
7   Aus   An   An   An   An  Aus
8    An  Aus  Aus  Aus  Aus  Aus
9    An  Aus  Aus   An  Aus  Aus
10   An  Aus   An  Aus  Aus   An
11   An  Aus   An   An  Aus  Aus
12   An   An  Aus  Aus  Aus  Aus
13   An   An  Aus   An  Aus  Aus
14   An   An   An  Aus  Aus   An
15   An   An   An   An  Aus  Aus

Ausgabe gespeichert in "output/nandu4.csv"
	\end{minted}
	      \newpage
	\item \textbf{nandu5.txt}\begin{minted}[breaklines]{text}
Simuliert in 15.92ms

     Q1   Q2   Q3   Q4   Q5   Q6   L1   L2   L3   L4   L5
0   Aus  Aus  Aus  Aus  Aus  Aus  Aus  Aus  Aus   An  Aus
1   Aus  Aus  Aus  Aus  Aus   An  Aus  Aus  Aus   An  Aus
2   Aus  Aus  Aus  Aus   An  Aus  Aus  Aus  Aus   An   An
3   Aus  Aus  Aus  Aus   An   An  Aus  Aus  Aus   An   An
4   Aus  Aus  Aus   An  Aus  Aus  Aus  Aus   An  Aus  Aus
..  ...  ...  ...  ...  ...  ...  ...  ...  ...  ...  ...
59   An   An   An  Aus   An   An   An  Aus  Aus   An   An
60   An   An   An   An  Aus  Aus   An  Aus   An  Aus  Aus
61   An   An   An   An  Aus   An   An  Aus   An  Aus  Aus
62   An   An   An   An   An  Aus   An  Aus  Aus   An   An
63   An   An   An   An   An   An   An  Aus  Aus   An   An

[64 rows x 11 columns]

Ausgabe gespeichert in "output/nandu5.csv"
			\end{minted}
	      \textbf{output/nandu5.csv}: \url{https://github.com/leonhma/bwinf-42-1/blob/main/a4-nandu/output/nandu5.csv}
\end{itemize}

\section{Quellcode}

\textit{program.py}
\inputminted[breaklines,linenos,fontsize=\footnotesize]{python3}{program.py}
\newpage

\end{document}