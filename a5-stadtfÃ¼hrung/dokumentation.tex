\documentclass[a4paper,12pt]{scrartcl}

\usepackage[english,ngerman]{babel}
\usepackage{csquotes}
\usepackage{color}
\usepackage{siunitx}
\usepackage{amssymb}
\usepackage{amsthm}
\usepackage{amsmath}
\usepackage{graphicx}
\usepackage{anyfontsize}
\usepackage[hidelinks]{hyperref}
\usepackage{subcaption}
\usepackage{enumitem}
\usepackage{tabu}
\usepackage{tabularx}
\usepackage[ngerman]{cleveref}
\usepackage[onehalfspacing]{setspace}
\usepackage[ backend=biber]{biblatex}
\addbibresource{refs.bib}

\title{\textbf{\Huge Aufgabe 5: „Stadtführung“}}
\author{\LARGE Team-ID: \LARGE 00128 \\\\
	    \LARGE Team-Name: \LARGE E29C8CF09F8E89 \\\\
	    \LARGE Bearbeiter/-innen dieser Aufgabe: \\ 
	    \LARGE Leonhard Masche\\\\}
\date{\LARGE\today}

% code
\usepackage[cache=false]{minted}
\usemintedstyle{xcode}

% set up geometry
% \usepackage[lmargin={3.5cm},rmargin={2cm}, tmargin={2cm},bmargin = {2cm}]{geometry}

% font
\usepackage{fontspec}
% \setmonofont{Fira Code SemiBold}
\setmainfont{Arial}

% configure headers and footers
\graphicspath{{./assets/}}

% custom commands
\newcommand*\wildcard[2][5cm]{\vspace*{2cm}\parbox{#1}{\hrulefill\par#2}}
\newcommand{\fig}[3][]{\begin{figure}[H] \includegraphics[#1]{#2}
\centering\caption{#3}\end{figure}}
\newcommand{\pyinline}[1]{\mintinline{python}{#1}}

\pagenumbering{arabic}

\theoremstyle{definition}
\newtheorem{theorem}{Theorem}[section]
\newtheorem{lemma}[theorem]{Lemma}

% hyphenation
\hyphenation{Fraun-hof-er}

% \setcounter{biburlnumpenalty}{100}
% \setcounter{biburlucpenalty}{100}
% \setcounter{biburllcpenalty}{100}

\usepackage[hang]{footmisc}
\setlength{\footnotemargin}{0.8em}

\begin{document}

\maketitle
\tableofcontents
\section{Lösungsidee}

Da der Startpunkt der Tour egal ist, und nur essentielle Orte („Knoten“)
bestehen bleiben müssen, werden alle nonessentiellen Knoten vom Anfang der Tour
entfernt. Nun werden alle Start- und End-Indices von Subtouren generiert. Dazu
wird von jedem Index in der Tour (linker Pointer) aus ein zweiter Pointer
inkrementiert, bis entweder: ein Knoten mit demselben Namen gefunden wurde (die
Subtour wird hinzugefügt), oder ein essentieller Knoten gefunden wurde. Aus
diesen Subtouren wird (entsprechend dem Weighted Interval Scheduling Problem)
die größtmögliche Reihenfolge von Subtouren ermittelt. Diese werden nun
entfernt, und das Ergebnis ausgegeben.

\section{Umsetzung}

Das Programm (\texttt{program.py}) ist in Python umgesetzt und mit einer
Umgebung ab der Version 3.8 ausführbar.

Beim Ausführen der Datei wird zuerst nach der Zahl des Beispiels gefragt. Dieses
wird nun aus der Datei \texttt{input/tour\{n\}.txt} geladen und bearbeitet. Die
Tour wird als chronologisch sortierte Liste gespeichert. Die Subtouren werden
(wie beschrieben im vorherigen Abschnitt) durch zwei verschachtelte
\pyinline{for}-Schleifen ermittelt. Durch dynamic programming wird die längste
Kombination von Subtouren ermittelt, um diese dann aus der Liste zu entfernen.
\footfullcite[Vgl.][]{mouatadidDynamicProgrammingWeighted} Zum Schluss wird die
gekürzte Tour in die Konsole ausgegeben. Das Programm läuft mit einer
Zeitkomplexität von $\mathcal{O}(n²)$ und alle Beispiele werden in unter 0.05ms
bearbeitet.

\section{Beispiele}

Hier wird das Programm auf die 5 Beispiele von der Website angewendet:

\begin{itemize}
	\item \textbf{tour1.txt}\begin{minted}[breaklines]{text}
Berechnet in 0.02ms

Tour:   (Gesamtlänge 1020)
1. Brauerei (1613)
2. Karzer (1665)
3. Rathaus (1678,1739)
4. Euler-Brücke (1768)
5. Fibonacci-Gaststätte (1820)
6. Schiefes Haus (1823)
7. Theater (1880)
8. Emmy-Noether-Campus (1912,1998)
9. Euler-Brücke (1999)
10. Brauerei (2012)
	\end{minted}
	\item \textbf{tour2.txt}\begin{minted}[breaklines]{text}
Berechnet in 0.03ms

Tour:   (Gesamtlänge 940)
1. Karzer (1665)
2. Rathaus (1678,1739)
3. Euler-Brücke (1768)
4. Fibonacci-Gaststätte (1820)
5. Schiefes Haus (1823)
6. Theater (1880)
7. Emmy-Noether-Campus (1912,1998)
8. Euler-Brücke (1999)
9. Brauerei (2012)
	\end{minted}
	\item \textbf{tour3.txt}\begin{minted}[breaklines]{text}
Berechnet in 0.02ms

Tour:   (Gesamtlänge 1220)
1. Observatorium (1874)
2. Piz Spitz (1898)
3. Panoramasteg (1912,1952)
4. Ziegenbrücke (1979)
5. Talstation (2005)
	\end{minted}
	\item \textbf{tour4.txt}\begin{minted}[breaklines]{text}
Berechnet in 0.03ms

Tour:   (Gesamtlänge 1640)
1. Dom (1596)
2. Bogenschütze (1610,1683)
3. Schnecke (1698)
4. Fischweiher (1710)
5. Reiterhof (1728)
6. Schnecke (1742)
7. Schmiede (1765)
8. Große Gabel (1794,1874)
9. Fingerhut (1917)
10. Stadion (1934)
11. Marktplatz (1962)
12. Baumschule (1974)
13. Polizeipräsidium (1991)
14. Blaues Pferd (2004)
	\end{minted}
	\item \textbf{tour5.txt}\begin{minted}[breaklines]{text}
Berechnet in 0.04ms

Tour:   (Gesamtlänge 2460)
1. Hexentanzplatz (1703)
2. Eselsbrücke (1711)
3. Dreibannstein (1724,1752)
4. Schmetterling (1760)
5. Dreibannstein (1781)
6. Märchenwald (1793,1840)
7. Eselsbrücke (1855,1877)
8. Reiterdenkmal (1880)
9. Riesenrad (1881,1902)
10. Dreibannstein (1911)
11. Olympisches Dorf (1924)
12. Haus der Zukunft (1927)
13. Stellwerk (1931,1942)
14. Labyrinth (1955)
15. Gauklerstadl (1961)
16. Planetarium (1971)
17. Känguruhfarm (1976)
18. Balzplatz (1978)
19. Dreibannstein (1998)
20. Labyrinth (2013)
21. CO2-Speicher (2022)
22. Gabelhaus (2023)
	\end{minted}
\end{itemize}

\section{Quellcode}

\textit{program.py}
\inputminted[breaklines,linenos,fontsize=\footnotesize]{python3}{program.py}
\newpage

\end{document}